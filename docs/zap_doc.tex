\documentclass[11pt]{article}

%%%%%%%%%%%%%%%%%%%%%%%%%%%%%%%%%%%%%%%%%%%%%%%%%%%%%%%%%%%%%%%%%%%%%%%%%%%%%%%

\usepackage{graphicx}
\usepackage{textcomp}
\usepackage{parskip}
\usepackage{mathptmx}
\usepackage[margin=1.0in]{geometry}
\usepackage{array}
\usepackage{tabularx}
\usepackage{minted}
\usepackage{multirow}
\usepackage{textcomp}
\pagestyle{headings}

%%%%%%%%%%%%%%%%%%%%%%%%%%%%%%%%%%%%%%%%%%%%%%%%%%%%%%%%%%%%%%%%%%%%%%%%%%%%%%%

\begin{document}

\title{ ZAP : An ARM v4T Compatible Soft Processor }
\author{Revanth Kamaraj}

\begin{titlepage}
\clearpage\maketitle
\thispagestyle{empty}
\maketitle
\end{titlepage}

%%%%%%%%%%%%%%%%%%%%%%%%%%%%%%%%%%%%%%%%%%%%%%%%%%%%%%%%%%%%%%%%%%%%%%%%%%%%%%%

\section{Introduction}

ZAP is an ARM{\textregistered} v4T compatible soft processor core. The code is
fully open source and is released under the MIT license.

\begin{center}
\textbf{MIT License} \\
\end{center}
Copyright (c) 2016 Revanth Kamaraj (Email: revanth91kamaraj@gmail.com) \\\\
Permission is hereby granted, free of charge, to any person obtaining a copy
of this software and associated documentation files (the "Software"), to deal
in the Software without restriction, including without limitation the rights
to use, copy, modify, merge, publish, distribute, sublicense, and/or sell
copies of the Software, and to permit persons to whom the Software is
furnished to do so, subject to the following conditions: \\\\
The above copyright notice and this permission notice shall be included in all
copies or substantial portions of the Software. \\\\
THE SOFTWARE IS PROVIDED "AS IS", WITHOUT WARRANTY OF ANY KIND, EXPRESS OR
IMPLIED, INCLUDING BUT NOT LIMITED TO THE WARRANTIES OF MERCHANTABILITY,
FITNESS FOR A PARTICULAR PURPOSE AND NONINFRINGEMENT. IN NO EVENT SHALL THE
AUTHORS OR COPYRIGHT HOLDERS BE LIABLE FOR ANY CLAIM, DAMAGES OR OTHER
LIABILITY, WHETHER IN AN ACTION OF CONTRACT, TORT OR OTHERWISE, ARISING FROM,
OUT OF OR IN CONNECTION WITH THE SOFTWARE OR THE USE OR OTHER DEALINGS IN THE
SOFTWARE.

%%%%%%%%%%%%%%%%%%%%%%%%%%%%%%%%%%%%%%%%%%%%%%%%%%%%%%%%%%%%%%%%%%%%%%%%%%%%%%%

\pagebreak
\subsection{Features}

\begin{itemize}

\item Fully ARM v4T compatible. Software compatible with the ARM7TM core. This
includes support for the 16-bit compressed instruction set.

\item The processor is super-pipelined with a 9 stage pipeline to achieve a high
operating frequency. The pipeline has a data forwarding capability to allow 
back to back instructions to execute without stalls. Non trivial shifts require
their operands a cycle early. Loads have a 3 cycle latency and the pipeline
will stall if an attempt is made to access the register within the latency
period.
\item Features dedicated high level sensitive IRQ, FIQ and memory abort pins.

\item Can be interfaced with caches/MMU.

\item The CPSR of the processor is exposed as a port allowing for 
implementation of a virtual memory system.

\item Features a coprocessor interface that allows the core to be extended. The
coprocessor interface simply exposes internal signals of the core. It is up to 
the coprocessor to interpret and process instructions correctly.

\item Supports both short and long multiply and multiply-accumulate operations.

\item The processor may be synthesized without compressed instruction support
and/or coprocessor interface support.

\item Most memory structures of the processor map efficiently onto FPGA block
RAMs. The register file is overclocked by a factor of 2.

\item Multiplication is performed using a 17x17 signed multiplier. This 
multiplier maps to dedicated silicon DSP blocks found on FPGAs.

\item No device specific instantiations are made to allow for portability
across FPGA vendors.

\item Having two write ports in the register file allows memory access with
writeback to issue at once instead of being split into two operations. This
may improve performance.

\item The processor core is written entirely in synthesizable Verilog-2001.

\item Features a dynamic branch predictor that is installed within the
pipeline to compensate for its length. Branches within a 2KB block of memory 
can be mapped into the predictor without conflict.

\item Uses a base restored abort model making it easier to write exception
handlers. Basically, on a fault in between a multiple memory transfer, the
processor rolls back the base pointer register.

\end{itemize}

%%%%%%%%%%%%%%%%%%%%%%%%%%%%%%%%%%%%%%%%%%%%%%%%%%%%%%%%%%%%%%%%%%%%%%%%%%%%%%%

\subsection{About This Manual}
The purpose of this manual is to document the processor core's design. This
document is very incomplete. I will try my best to update it.

%%%%%%%%%%%%%%%%%%%%%%%%%%%%%%%%%%%%%%%%%%%%%%%%%%%%%%%%%%%%%%%%%%%%%%%%%%%%%%%
\pagebreak
\section{Configuring the core and testbench}
Throughout, it is assumed that \texttt{\$ZAP\_HOME} points to the working
directory of the project.\\
Core/testbench configuration may be done using defines. The defines file is 
located in \\
\texttt{\$ZAP\_HOME/includes/config.vh}. \\

\begin{tabularx}{\textwidth}{| X | X | X | X |} 
\hline
Define & Purpose & Required for & Comments \\  
\hline
\multicolumn{4}{c}{CORE CONFIGURATION} \\
\hline
\texttt{THUMB\_EN} & Enabling compressed instruction support & Core Setup & 
Enabling this increases core area and reduces performance. \\ 
\texttt{COPROC\_IF\_EN} & Enabling coprocessor suport. Extra ports get added. &
 Core Setup & Enabling this increases core area and reduces performance. \\
\hline
\multicolumn{4}{c}{TESTBENCH CONFIGURATION} \\
\hline
\texttt{IRQ\_EN} & Generates periodic IRQ pulses. & Testbench & -- \\
\texttt{SIM} & Generates extra messages. & Testbench & \emph{Must be UNDEFINED 
for correct synthesis in Xilinx.} \\
\texttt{VCD\_FILE\_PATH} & Set the path to the VCD data dump. & Testbench & -- \\
\texttt{MEMORY\_IMAGE} & Path to the memory image Verilog file. & Testbench & 
-- \\
\texttt{MAX\_CLOCK\_CYCLES} & Set the number of cycles the simulation should 
run before terminating. & Testbench & -- \\
\texttt{SEED} & Set the testbench seed. The seed influences randomness. & 
Testbench & -- \\
\hline
\end{tabularx}

%%%%%%%%%%%%%%%%%%%%%%%%%%%%%%%%%%%%%%%%%%%%%%%%%%%%%%%%%%%%%%%%%%%%%%%%%%%%%%%

\pagebreak
\section{Compiling Code and Running Simulation}
\subsubsection{Generating a binary using GNU tools}
You can use the existing GNU toolchain to generate code for the processor. This
section will briefly explain the procedure. For the purposes of this 
discussion, let us assume these are the source files...\\
\texttt{main.c} \\
\texttt{fact.c} \\
\texttt{startup.s} \\
\texttt{misc.s} \\
\texttt{linker.ld}  \emph{ This is the linker script.} \\\\
Generate a bunch of object files.\\
\texttt{arm-none-eabi-as -mcpu=arm7tdmi -g startup.s -o startup.o} \\
\texttt{arm-none-eabi-as -mcpu=arm7tdmi -g misc.s -o misc.o} \\
\texttt{arm-none-eabi-gcc -c -mcpu=arm7tdmi -g main.c -o main.o} \\
\texttt{arm-none-eabi-gcc -c -mcpu=arm7tdmi -g fact.c -o fact.o} \\\\
Link them up using a linker script...\\
\texttt{arm-none-eabi-ld -T linker.ld startup.o misc.o main.o fact.o -o prog.elf} \\\\
Finally generate a flat binary...\\
\texttt{arm-none-eabi-objcopy -O binary prog.elf prog.bin} \\\\
The .bin file generated is the flat binary.\\\\

%%%%%%%%%%%%%%%%%%%%%%%%%%%%%%%%%%%%%%%%%%%%%%%%%%%%%%%%%%%%%%%%%%%%%%%%%%%%%%%

\subsubsection{Generating a Verilog memory map}
\texttt{perl \$ZAP\_HOME/scripts/bin2mem.pl prog.bin prog.v}\\
The prog.v file looks like this...\\\\
\texttt{mem[0] = 8'b00;} \\
\texttt{mem[1] = 8'b01;} \\\\

%%%%%%%%%%%%%%%%%%%%%%%%%%%%%%%%%%%%%%%%%%%%%%%%%%%%%%%%%%%%%%%%%%%%%%%%%%%%%%%

\subsubsection{Invoking the simulator}
Ensure \texttt{config.vh} is set up correctly.

Your command must look like this...\\\\
\texttt{iverilog \$ZAP\_HOME/rtl/*.v \$ZAP\_HOME/*.v \$ZAP\_HOME/testbench/*.v
\$ZAP\_HOME/models/ram/ram.v -DSEED=22}\\\\

The rtl/*.v and rtl/*/*.v collect all of the synthesizable Verilog-2001 files,
the testbench/*.v collects all of the testbench (In this situations, the ram.v
file is a part of the testbench).

Provide some seed value (22 is used in the example). Ensure you edit the 
\texttt{config.vh} file before running the simulation to correctly point to the
memory map, vcd target output path etc for the simulator to pick up.

%%%%%%%%%%%%%%%%%%%%%%%%%%%%%%%%%%%%%%%%%%%%%%%%%%%%%%%%%%%%%%%%%%%%%%%%%%%%%%%

\subsection{Run Sample Code quickly...}
A sample .s and .c file is present in \texttt{\$ZAP\_HOME/sw/s} and 
\texttt{\$ZAP\_HOME/sw/c} respectively. To translate them to binary and to a 
Verilog memory map, you can run the Perl script \\\\ 
\texttt{perl \$ZAP\_HOME/debug/run\_sim.pl} \\\\ Ensure you set all the variables
in the script as per the table below...\\\\

% This table tells what to place in script variables.
\begin{tabularx}{\textwidth}{| X | X |}
\hline
Variable & Purpose \\
\hline
\texttt{ZAP\_HOME} & Set this to the project working directory.\\\\
\texttt{LOG\_FILE\_PATH} & Set this to the place where you want the log file to 
be created.\\\\
\texttt{ASM\_PATH} & Set this to the location of your startup assembly file.
\\\\
\texttt{C\_PATH} & Set this to the location of your C file. \\\\
\texttt{LINKER\_PATH} & Set this to the location of the linker script. \\\\
\texttt{TARGET\_BIN\_PATH} & Set this to the target bin file location. \\\\
\texttt{VCD\_PATH} & Set this to location where the VCD is to be created. 
\textbf{This must match what is in config.vh} \\\\
\texttt{MEMORY\_IMAGE} & Set this to the location where the memory image must be
created. \textbf{This must match what is in config.vh} \\\\
\hline
\end{tabularx}

%%%%%%%%%%%%%%%%%%%%%%%%%%%%%%%%%%%%%%%%%%%%%%%%%%%%%%%%%%%%%%%%%%%%%%%%%%%%%%%

\end{document}

